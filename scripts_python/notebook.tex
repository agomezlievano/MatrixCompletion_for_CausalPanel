
% Default to the notebook output style

    


% Inherit from the specified cell style.




    
\documentclass[11pt]{article}

    
    
    \usepackage[T1]{fontenc}
    % Nicer default font (+ math font) than Computer Modern for most use cases
    \usepackage{mathpazo}

    % Basic figure setup, for now with no caption control since it's done
    % automatically by Pandoc (which extracts ![](path) syntax from Markdown).
    \usepackage{graphicx}
    % We will generate all images so they have a width \maxwidth. This means
    % that they will get their normal width if they fit onto the page, but
    % are scaled down if they would overflow the margins.
    \makeatletter
    \def\maxwidth{\ifdim\Gin@nat@width>\linewidth\linewidth
    \else\Gin@nat@width\fi}
    \makeatother
    \let\Oldincludegraphics\includegraphics
    % Set max figure width to be 80% of text width, for now hardcoded.
    \renewcommand{\includegraphics}[1]{\Oldincludegraphics[width=.8\maxwidth]{#1}}
    % Ensure that by default, figures have no caption (until we provide a
    % proper Figure object with a Caption API and a way to capture that
    % in the conversion process - todo).
    \usepackage{caption}
    \DeclareCaptionLabelFormat{nolabel}{}
    \captionsetup{labelformat=nolabel}

    \usepackage{adjustbox} % Used to constrain images to a maximum size 
    \usepackage{xcolor} % Allow colors to be defined
    \usepackage{enumerate} % Needed for markdown enumerations to work
    \usepackage{geometry} % Used to adjust the document margins
    \usepackage{amsmath} % Equations
    \usepackage{amssymb} % Equations
    \usepackage{textcomp} % defines textquotesingle
    % Hack from http://tex.stackexchange.com/a/47451/13684:
    \AtBeginDocument{%
        \def\PYZsq{\textquotesingle}% Upright quotes in Pygmentized code
    }
    \usepackage{upquote} % Upright quotes for verbatim code
    \usepackage{eurosym} % defines \euro
    \usepackage[mathletters]{ucs} % Extended unicode (utf-8) support
    \usepackage[utf8x]{inputenc} % Allow utf-8 characters in the tex document
    \usepackage{fancyvrb} % verbatim replacement that allows latex
    \usepackage{grffile} % extends the file name processing of package graphics 
                         % to support a larger range 
    % The hyperref package gives us a pdf with properly built
    % internal navigation ('pdf bookmarks' for the table of contents,
    % internal cross-reference links, web links for URLs, etc.)
    \usepackage{hyperref}
    \usepackage{longtable} % longtable support required by pandoc >1.10
    \usepackage{booktabs}  % table support for pandoc > 1.12.2
    \usepackage[inline]{enumitem} % IRkernel/repr support (it uses the enumerate* environment)
    \usepackage[normalem]{ulem} % ulem is needed to support strikethroughs (\sout)
                                % normalem makes italics be italics, not underlines
    

    
    
    % Colors for the hyperref package
    \definecolor{urlcolor}{rgb}{0,.145,.698}
    \definecolor{linkcolor}{rgb}{.71,0.21,0.01}
    \definecolor{citecolor}{rgb}{.12,.54,.11}

    % ANSI colors
    \definecolor{ansi-black}{HTML}{3E424D}
    \definecolor{ansi-black-intense}{HTML}{282C36}
    \definecolor{ansi-red}{HTML}{E75C58}
    \definecolor{ansi-red-intense}{HTML}{B22B31}
    \definecolor{ansi-green}{HTML}{00A250}
    \definecolor{ansi-green-intense}{HTML}{007427}
    \definecolor{ansi-yellow}{HTML}{DDB62B}
    \definecolor{ansi-yellow-intense}{HTML}{B27D12}
    \definecolor{ansi-blue}{HTML}{208FFB}
    \definecolor{ansi-blue-intense}{HTML}{0065CA}
    \definecolor{ansi-magenta}{HTML}{D160C4}
    \definecolor{ansi-magenta-intense}{HTML}{A03196}
    \definecolor{ansi-cyan}{HTML}{60C6C8}
    \definecolor{ansi-cyan-intense}{HTML}{258F8F}
    \definecolor{ansi-white}{HTML}{C5C1B4}
    \definecolor{ansi-white-intense}{HTML}{A1A6B2}

    % commands and environments needed by pandoc snippets
    % extracted from the output of `pandoc -s`
    \providecommand{\tightlist}{%
      \setlength{\itemsep}{0pt}\setlength{\parskip}{0pt}}
    \DefineVerbatimEnvironment{Highlighting}{Verbatim}{commandchars=\\\{\}}
    % Add ',fontsize=\small' for more characters per line
    \newenvironment{Shaded}{}{}
    \newcommand{\KeywordTok}[1]{\textcolor[rgb]{0.00,0.44,0.13}{\textbf{{#1}}}}
    \newcommand{\DataTypeTok}[1]{\textcolor[rgb]{0.56,0.13,0.00}{{#1}}}
    \newcommand{\DecValTok}[1]{\textcolor[rgb]{0.25,0.63,0.44}{{#1}}}
    \newcommand{\BaseNTok}[1]{\textcolor[rgb]{0.25,0.63,0.44}{{#1}}}
    \newcommand{\FloatTok}[1]{\textcolor[rgb]{0.25,0.63,0.44}{{#1}}}
    \newcommand{\CharTok}[1]{\textcolor[rgb]{0.25,0.44,0.63}{{#1}}}
    \newcommand{\StringTok}[1]{\textcolor[rgb]{0.25,0.44,0.63}{{#1}}}
    \newcommand{\CommentTok}[1]{\textcolor[rgb]{0.38,0.63,0.69}{\textit{{#1}}}}
    \newcommand{\OtherTok}[1]{\textcolor[rgb]{0.00,0.44,0.13}{{#1}}}
    \newcommand{\AlertTok}[1]{\textcolor[rgb]{1.00,0.00,0.00}{\textbf{{#1}}}}
    \newcommand{\FunctionTok}[1]{\textcolor[rgb]{0.02,0.16,0.49}{{#1}}}
    \newcommand{\RegionMarkerTok}[1]{{#1}}
    \newcommand{\ErrorTok}[1]{\textcolor[rgb]{1.00,0.00,0.00}{\textbf{{#1}}}}
    \newcommand{\NormalTok}[1]{{#1}}
    
    % Additional commands for more recent versions of Pandoc
    \newcommand{\ConstantTok}[1]{\textcolor[rgb]{0.53,0.00,0.00}{{#1}}}
    \newcommand{\SpecialCharTok}[1]{\textcolor[rgb]{0.25,0.44,0.63}{{#1}}}
    \newcommand{\VerbatimStringTok}[1]{\textcolor[rgb]{0.25,0.44,0.63}{{#1}}}
    \newcommand{\SpecialStringTok}[1]{\textcolor[rgb]{0.73,0.40,0.53}{{#1}}}
    \newcommand{\ImportTok}[1]{{#1}}
    \newcommand{\DocumentationTok}[1]{\textcolor[rgb]{0.73,0.13,0.13}{\textit{{#1}}}}
    \newcommand{\AnnotationTok}[1]{\textcolor[rgb]{0.38,0.63,0.69}{\textbf{\textit{{#1}}}}}
    \newcommand{\CommentVarTok}[1]{\textcolor[rgb]{0.38,0.63,0.69}{\textbf{\textit{{#1}}}}}
    \newcommand{\VariableTok}[1]{\textcolor[rgb]{0.10,0.09,0.49}{{#1}}}
    \newcommand{\ControlFlowTok}[1]{\textcolor[rgb]{0.00,0.44,0.13}{\textbf{{#1}}}}
    \newcommand{\OperatorTok}[1]{\textcolor[rgb]{0.40,0.40,0.40}{{#1}}}
    \newcommand{\BuiltInTok}[1]{{#1}}
    \newcommand{\ExtensionTok}[1]{{#1}}
    \newcommand{\PreprocessorTok}[1]{\textcolor[rgb]{0.74,0.48,0.00}{{#1}}}
    \newcommand{\AttributeTok}[1]{\textcolor[rgb]{0.49,0.56,0.16}{{#1}}}
    \newcommand{\InformationTok}[1]{\textcolor[rgb]{0.38,0.63,0.69}{\textbf{\textit{{#1}}}}}
    \newcommand{\WarningTok}[1]{\textcolor[rgb]{0.38,0.63,0.69}{\textbf{\textit{{#1}}}}}
    
    
    % Define a nice break command that doesn't care if a line doesn't already
    % exist.
    \def\br{\hspace*{\fill} \\* }
    % Math Jax compatability definitions
    \def\gt{>}
    \def\lt{<}
    % Document parameters
    \title{AtheyMatCompletion\_0001\_loading\_and\_exploring}
    
    
    

    % Pygments definitions
    
\makeatletter
\def\PY@reset{\let\PY@it=\relax \let\PY@bf=\relax%
    \let\PY@ul=\relax \let\PY@tc=\relax%
    \let\PY@bc=\relax \let\PY@ff=\relax}
\def\PY@tok#1{\csname PY@tok@#1\endcsname}
\def\PY@toks#1+{\ifx\relax#1\empty\else%
    \PY@tok{#1}\expandafter\PY@toks\fi}
\def\PY@do#1{\PY@bc{\PY@tc{\PY@ul{%
    \PY@it{\PY@bf{\PY@ff{#1}}}}}}}
\def\PY#1#2{\PY@reset\PY@toks#1+\relax+\PY@do{#2}}

\expandafter\def\csname PY@tok@w\endcsname{\def\PY@tc##1{\textcolor[rgb]{0.73,0.73,0.73}{##1}}}
\expandafter\def\csname PY@tok@c\endcsname{\let\PY@it=\textit\def\PY@tc##1{\textcolor[rgb]{0.25,0.50,0.50}{##1}}}
\expandafter\def\csname PY@tok@cp\endcsname{\def\PY@tc##1{\textcolor[rgb]{0.74,0.48,0.00}{##1}}}
\expandafter\def\csname PY@tok@k\endcsname{\let\PY@bf=\textbf\def\PY@tc##1{\textcolor[rgb]{0.00,0.50,0.00}{##1}}}
\expandafter\def\csname PY@tok@kp\endcsname{\def\PY@tc##1{\textcolor[rgb]{0.00,0.50,0.00}{##1}}}
\expandafter\def\csname PY@tok@kt\endcsname{\def\PY@tc##1{\textcolor[rgb]{0.69,0.00,0.25}{##1}}}
\expandafter\def\csname PY@tok@o\endcsname{\def\PY@tc##1{\textcolor[rgb]{0.40,0.40,0.40}{##1}}}
\expandafter\def\csname PY@tok@ow\endcsname{\let\PY@bf=\textbf\def\PY@tc##1{\textcolor[rgb]{0.67,0.13,1.00}{##1}}}
\expandafter\def\csname PY@tok@nb\endcsname{\def\PY@tc##1{\textcolor[rgb]{0.00,0.50,0.00}{##1}}}
\expandafter\def\csname PY@tok@nf\endcsname{\def\PY@tc##1{\textcolor[rgb]{0.00,0.00,1.00}{##1}}}
\expandafter\def\csname PY@tok@nc\endcsname{\let\PY@bf=\textbf\def\PY@tc##1{\textcolor[rgb]{0.00,0.00,1.00}{##1}}}
\expandafter\def\csname PY@tok@nn\endcsname{\let\PY@bf=\textbf\def\PY@tc##1{\textcolor[rgb]{0.00,0.00,1.00}{##1}}}
\expandafter\def\csname PY@tok@ne\endcsname{\let\PY@bf=\textbf\def\PY@tc##1{\textcolor[rgb]{0.82,0.25,0.23}{##1}}}
\expandafter\def\csname PY@tok@nv\endcsname{\def\PY@tc##1{\textcolor[rgb]{0.10,0.09,0.49}{##1}}}
\expandafter\def\csname PY@tok@no\endcsname{\def\PY@tc##1{\textcolor[rgb]{0.53,0.00,0.00}{##1}}}
\expandafter\def\csname PY@tok@nl\endcsname{\def\PY@tc##1{\textcolor[rgb]{0.63,0.63,0.00}{##1}}}
\expandafter\def\csname PY@tok@ni\endcsname{\let\PY@bf=\textbf\def\PY@tc##1{\textcolor[rgb]{0.60,0.60,0.60}{##1}}}
\expandafter\def\csname PY@tok@na\endcsname{\def\PY@tc##1{\textcolor[rgb]{0.49,0.56,0.16}{##1}}}
\expandafter\def\csname PY@tok@nt\endcsname{\let\PY@bf=\textbf\def\PY@tc##1{\textcolor[rgb]{0.00,0.50,0.00}{##1}}}
\expandafter\def\csname PY@tok@nd\endcsname{\def\PY@tc##1{\textcolor[rgb]{0.67,0.13,1.00}{##1}}}
\expandafter\def\csname PY@tok@s\endcsname{\def\PY@tc##1{\textcolor[rgb]{0.73,0.13,0.13}{##1}}}
\expandafter\def\csname PY@tok@sd\endcsname{\let\PY@it=\textit\def\PY@tc##1{\textcolor[rgb]{0.73,0.13,0.13}{##1}}}
\expandafter\def\csname PY@tok@si\endcsname{\let\PY@bf=\textbf\def\PY@tc##1{\textcolor[rgb]{0.73,0.40,0.53}{##1}}}
\expandafter\def\csname PY@tok@se\endcsname{\let\PY@bf=\textbf\def\PY@tc##1{\textcolor[rgb]{0.73,0.40,0.13}{##1}}}
\expandafter\def\csname PY@tok@sr\endcsname{\def\PY@tc##1{\textcolor[rgb]{0.73,0.40,0.53}{##1}}}
\expandafter\def\csname PY@tok@ss\endcsname{\def\PY@tc##1{\textcolor[rgb]{0.10,0.09,0.49}{##1}}}
\expandafter\def\csname PY@tok@sx\endcsname{\def\PY@tc##1{\textcolor[rgb]{0.00,0.50,0.00}{##1}}}
\expandafter\def\csname PY@tok@m\endcsname{\def\PY@tc##1{\textcolor[rgb]{0.40,0.40,0.40}{##1}}}
\expandafter\def\csname PY@tok@gh\endcsname{\let\PY@bf=\textbf\def\PY@tc##1{\textcolor[rgb]{0.00,0.00,0.50}{##1}}}
\expandafter\def\csname PY@tok@gu\endcsname{\let\PY@bf=\textbf\def\PY@tc##1{\textcolor[rgb]{0.50,0.00,0.50}{##1}}}
\expandafter\def\csname PY@tok@gd\endcsname{\def\PY@tc##1{\textcolor[rgb]{0.63,0.00,0.00}{##1}}}
\expandafter\def\csname PY@tok@gi\endcsname{\def\PY@tc##1{\textcolor[rgb]{0.00,0.63,0.00}{##1}}}
\expandafter\def\csname PY@tok@gr\endcsname{\def\PY@tc##1{\textcolor[rgb]{1.00,0.00,0.00}{##1}}}
\expandafter\def\csname PY@tok@ge\endcsname{\let\PY@it=\textit}
\expandafter\def\csname PY@tok@gs\endcsname{\let\PY@bf=\textbf}
\expandafter\def\csname PY@tok@gp\endcsname{\let\PY@bf=\textbf\def\PY@tc##1{\textcolor[rgb]{0.00,0.00,0.50}{##1}}}
\expandafter\def\csname PY@tok@go\endcsname{\def\PY@tc##1{\textcolor[rgb]{0.53,0.53,0.53}{##1}}}
\expandafter\def\csname PY@tok@gt\endcsname{\def\PY@tc##1{\textcolor[rgb]{0.00,0.27,0.87}{##1}}}
\expandafter\def\csname PY@tok@err\endcsname{\def\PY@bc##1{\setlength{\fboxsep}{0pt}\fcolorbox[rgb]{1.00,0.00,0.00}{1,1,1}{\strut ##1}}}
\expandafter\def\csname PY@tok@kc\endcsname{\let\PY@bf=\textbf\def\PY@tc##1{\textcolor[rgb]{0.00,0.50,0.00}{##1}}}
\expandafter\def\csname PY@tok@kd\endcsname{\let\PY@bf=\textbf\def\PY@tc##1{\textcolor[rgb]{0.00,0.50,0.00}{##1}}}
\expandafter\def\csname PY@tok@kn\endcsname{\let\PY@bf=\textbf\def\PY@tc##1{\textcolor[rgb]{0.00,0.50,0.00}{##1}}}
\expandafter\def\csname PY@tok@kr\endcsname{\let\PY@bf=\textbf\def\PY@tc##1{\textcolor[rgb]{0.00,0.50,0.00}{##1}}}
\expandafter\def\csname PY@tok@bp\endcsname{\def\PY@tc##1{\textcolor[rgb]{0.00,0.50,0.00}{##1}}}
\expandafter\def\csname PY@tok@fm\endcsname{\def\PY@tc##1{\textcolor[rgb]{0.00,0.00,1.00}{##1}}}
\expandafter\def\csname PY@tok@vc\endcsname{\def\PY@tc##1{\textcolor[rgb]{0.10,0.09,0.49}{##1}}}
\expandafter\def\csname PY@tok@vg\endcsname{\def\PY@tc##1{\textcolor[rgb]{0.10,0.09,0.49}{##1}}}
\expandafter\def\csname PY@tok@vi\endcsname{\def\PY@tc##1{\textcolor[rgb]{0.10,0.09,0.49}{##1}}}
\expandafter\def\csname PY@tok@vm\endcsname{\def\PY@tc##1{\textcolor[rgb]{0.10,0.09,0.49}{##1}}}
\expandafter\def\csname PY@tok@sa\endcsname{\def\PY@tc##1{\textcolor[rgb]{0.73,0.13,0.13}{##1}}}
\expandafter\def\csname PY@tok@sb\endcsname{\def\PY@tc##1{\textcolor[rgb]{0.73,0.13,0.13}{##1}}}
\expandafter\def\csname PY@tok@sc\endcsname{\def\PY@tc##1{\textcolor[rgb]{0.73,0.13,0.13}{##1}}}
\expandafter\def\csname PY@tok@dl\endcsname{\def\PY@tc##1{\textcolor[rgb]{0.73,0.13,0.13}{##1}}}
\expandafter\def\csname PY@tok@s2\endcsname{\def\PY@tc##1{\textcolor[rgb]{0.73,0.13,0.13}{##1}}}
\expandafter\def\csname PY@tok@sh\endcsname{\def\PY@tc##1{\textcolor[rgb]{0.73,0.13,0.13}{##1}}}
\expandafter\def\csname PY@tok@s1\endcsname{\def\PY@tc##1{\textcolor[rgb]{0.73,0.13,0.13}{##1}}}
\expandafter\def\csname PY@tok@mb\endcsname{\def\PY@tc##1{\textcolor[rgb]{0.40,0.40,0.40}{##1}}}
\expandafter\def\csname PY@tok@mf\endcsname{\def\PY@tc##1{\textcolor[rgb]{0.40,0.40,0.40}{##1}}}
\expandafter\def\csname PY@tok@mh\endcsname{\def\PY@tc##1{\textcolor[rgb]{0.40,0.40,0.40}{##1}}}
\expandafter\def\csname PY@tok@mi\endcsname{\def\PY@tc##1{\textcolor[rgb]{0.40,0.40,0.40}{##1}}}
\expandafter\def\csname PY@tok@il\endcsname{\def\PY@tc##1{\textcolor[rgb]{0.40,0.40,0.40}{##1}}}
\expandafter\def\csname PY@tok@mo\endcsname{\def\PY@tc##1{\textcolor[rgb]{0.40,0.40,0.40}{##1}}}
\expandafter\def\csname PY@tok@ch\endcsname{\let\PY@it=\textit\def\PY@tc##1{\textcolor[rgb]{0.25,0.50,0.50}{##1}}}
\expandafter\def\csname PY@tok@cm\endcsname{\let\PY@it=\textit\def\PY@tc##1{\textcolor[rgb]{0.25,0.50,0.50}{##1}}}
\expandafter\def\csname PY@tok@cpf\endcsname{\let\PY@it=\textit\def\PY@tc##1{\textcolor[rgb]{0.25,0.50,0.50}{##1}}}
\expandafter\def\csname PY@tok@c1\endcsname{\let\PY@it=\textit\def\PY@tc##1{\textcolor[rgb]{0.25,0.50,0.50}{##1}}}
\expandafter\def\csname PY@tok@cs\endcsname{\let\PY@it=\textit\def\PY@tc##1{\textcolor[rgb]{0.25,0.50,0.50}{##1}}}

\def\PYZbs{\char`\\}
\def\PYZus{\char`\_}
\def\PYZob{\char`\{}
\def\PYZcb{\char`\}}
\def\PYZca{\char`\^}
\def\PYZam{\char`\&}
\def\PYZlt{\char`\<}
\def\PYZgt{\char`\>}
\def\PYZsh{\char`\#}
\def\PYZpc{\char`\%}
\def\PYZdl{\char`\$}
\def\PYZhy{\char`\-}
\def\PYZsq{\char`\'}
\def\PYZdq{\char`\"}
\def\PYZti{\char`\~}
% for compatibility with earlier versions
\def\PYZat{@}
\def\PYZlb{[}
\def\PYZrb{]}
\makeatother


    % Exact colors from NB
    \definecolor{incolor}{rgb}{0.0, 0.0, 0.5}
    \definecolor{outcolor}{rgb}{0.545, 0.0, 0.0}



    
    % Prevent overflowing lines due to hard-to-break entities
    \sloppy 
    % Setup hyperref package
    \hypersetup{
      breaklinks=true,  % so long urls are correctly broken across lines
      colorlinks=true,
      urlcolor=urlcolor,
      linkcolor=linkcolor,
      citecolor=citecolor,
      }
    % Slightly bigger margins than the latex defaults
    
    \geometry{verbose,tmargin=1in,bmargin=1in,lmargin=1in,rmargin=1in}
    
    

    \begin{document}
    
    
    \maketitle
    
    

    
    \begin{Verbatim}[commandchars=\\\{\}]
{\color{incolor}In [{\color{incolor}1}]:} \PY{o}{\PYZpc{}}\PY{k}{load\PYZus{}ext} autoreload
        \PY{o}{\PYZpc{}}\PY{k}{autoreload} 2
        \PY{o}{\PYZpc{}}\PY{k}{matplotlib} inline
\end{Verbatim}


    \begin{Verbatim}[commandchars=\\\{\}]
{\color{incolor}In [{\color{incolor}162}]:} \PY{k+kn}{import} \PY{n+nn}{time}
          \PY{k+kn}{import} \PY{n+nn}{datetime}
          
          \PY{k+kn}{import} \PY{n+nn}{pandas} \PY{k}{as} \PY{n+nn}{pd}
          \PY{k}{def} \PY{n+nf}{ends}\PY{p}{(}\PY{n}{df}\PY{p}{,} \PY{n}{x}\PY{o}{=}\PY{l+m+mi}{5}\PY{p}{)}\PY{p}{:}
              \PY{k}{return} \PY{n}{df}\PY{o}{.}\PY{n}{head}\PY{p}{(}\PY{n}{x}\PY{p}{)}\PY{o}{.}\PY{n}{append}\PY{p}{(}\PY{n}{df}\PY{o}{.}\PY{n}{tail}\PY{p}{(}\PY{n}{x}\PY{p}{)}\PY{p}{)}
          \PY{n+nb}{setattr}\PY{p}{(}\PY{n}{pd}\PY{o}{.}\PY{n}{DataFrame}\PY{p}{,}\PY{l+s+s1}{\PYZsq{}}\PY{l+s+s1}{ends}\PY{l+s+s1}{\PYZsq{}}\PY{p}{,}\PY{n}{ends}\PY{p}{)}
          \PY{k+kn}{import} \PY{n+nn}{numpy} \PY{k}{as} \PY{n+nn}{np}
          
          \PY{k+kn}{import} \PY{n+nn}{matplotlib}\PY{n+nn}{.}\PY{n+nn}{pyplot} \PY{k}{as} \PY{n+nn}{plt}
          \PY{k+kn}{import} \PY{n+nn}{seaborn} \PY{k}{as} \PY{n+nn}{sns}
          
          \PY{k+kn}{import} \PY{n+nn}{itertools}
          \PY{k+kn}{import} \PY{n+nn}{collections}
          
          \PY{k+kn}{from} \PY{n+nn}{sklearn}\PY{n+nn}{.}\PY{n+nn}{model\PYZus{}selection} \PY{k}{import} \PY{n}{KFold}
\end{Verbatim}


    \hypertarget{what-were-doing}{%
\section{What we're doing}\label{what-were-doing}}

    I have been very interested lately in two related, but different
questions: 1) How to \emph{apply} Machine Learning to Economics, and how
to integrate ML into the usual econometric pipeline? 2) What do the ML
algorithms themselves teach us about how economies work.

This notebook is about the point 1. In particular, about the problem of
estimating causal inference.

    We are going to replicate the methodology (and hopefully results) of a
recent paper in which the problem of causal inference is seen from the
lens of the problem of `'matrix completion'' from the machine learning
literature. I found out about this paper thanks to my dear friend Sid
Ravinutala.

    In the last months, I've become increasingly interested in Susan Athey's
work, who's precisely preaching about the importance of using ML in
economics.

    \hypertarget{set-up-of-the-problem}{%
\section{Set up of the problem}\label{set-up-of-the-problem}}

    The paper we want to reproduce is:

`'Matrix Completion Methods for Causal Panel Data Models'' by Susan
Athey, Mohsen Bayati, Nikolay Doudchenko, Guido Imbens, Khashayar
Khosravi (NBER Working Paper No.~25132, also available in
\href{https://arxiv.org/pdf/1710.10251.pdf}{arxiv})

    The setup of the problem is to consider \(N\) units across \(T\) time
steps. For each unit and time, there are two potential outcomes:
\(Y_{it}(0)\) if not treated, \(Y_{it}(1)\) if treated. The treatment
variable can be denoted by \(W_{it}\). Hence, the realized outcome is
\(Y_{it}=Y_{it}(W_{it})\).

The basic goal of causal identification is to estimate
\[\widehat{ATE} \equiv \frac{1}{NT}\sum_{i,t}\left(Y_{it}(1) - Y_{it}(0)\right.)\]
The fundamental problem in causal identification, however, is that we
never observe both \(Y_{it}(0)\) and \(Y_{it}(1)\). We either observe
one, or the other.

    Athey's et al. (2018) solution is based on the following: First,
\(Y_{it}(0)\) (or \(Y_{it}(1)\)) is a \(N\times T\) matrix with missing
values. Second, there are several methodologies to impute missing values
in matrices. Hence, to get the `'counterfactual'' matrix, we can make
use of the methods to impute missing values.

Athey et al.~focus specifically on imputing missing values in
\(Y_{it}(0)\) (i.e., the counterfactuals of those treated).

Many algorithms for imputing missing values in matrices are based on
matrix factorization methods. The different methods differ in the
constraints. See \href{https://arxiv.org/pdf/1410.0342.pdf}{Udell et al.
(2015)} for a good review.

    Athey's et al. (2018) use the method `'Nuclear Norm Matrix Completion
Estimator''.

    \hypertarget{nuclear-norm-matrix-completion-estimator}{%
\section{Nuclear Norm Matrix Completion
Estimator}\label{nuclear-norm-matrix-completion-estimator}}

    Let us represent the \(N\times T\) matrix we are interested in with
\(\mathbf{Y}=\mathbf{L^*}+\boldsymbol{\varepsilon}\), where
\(\boldsymbol{\varepsilon}\) can be considered as measurement error.
Athey et al.~set up the methodological problem as estimating \[
\widehat{\mathbf{L}}=\mathrm{arg min}_{\mathbf{L}}\left\{\frac{1}{\left|\mathcal{O}\right|}\left\| \mathbf{P}_{\mathcal{O}}(\mathbf{Y}-\mathbf{L}) \right\|_{F}^2 + \lambda \left\| \mathbf{L} \right\|_{*} \right\}.
\]

\begin{itemize}
\tightlist
\item
  The set of \emph{non-missing} elements is defined by all
  \((i,t)\in\mathcal{O}\), while the set of \emph{missing} elements by
  \((i,t)\in\mathcal{M}\).
\item
  The term \(\mathbf{P}_{\mathcal{O}}(\mathbf{A})\) is an operator that
  sets to \(0\) all the elements in matrix \(\mathbf{A}\) which do not
  belong to the set of matrix elements \(\mathcal{O}\). In other words,
  we just keep the non-missing elements \(\mathcal{O}\), but we replace
  the missing elements with 0's.
  \(\mathbf{P}_{\mathcal{O}}^{\bot}(\mathbf{A})\) would do the opposite.
\item
  The term \(\left\| \cdot \right\|_{F}^2\) is the Frobenius Norm:
  \(\left\| \mathbf{A} \right\|_{F}^2 = \sum_{i,j}A_{i,j}^2=\sum_k\sigma_k(\mathbf{A})^2\),
  where \(\sigma_i(\mathbf{A})\) are the singular values of the matrix
  (given by a Singular Value Decomposition).
\item
  The term \(\left\| \cdot \right\|_{*}\) is the \textbf{Nuclear Norm}
  regularization of the minimization problem, given by
  \(\left\| \mathbf{A} \right\|_{*} = \sum_k\sigma_k(\mathbf{A})\).
  Recall that the singular values are always non-negative.
\end{itemize}

    The regularization has a very important role in the problem, since we
want \(\mathbf{L}\) to approximate the matrix \(\mathbf{Y}\), not make
it equal, but only taking into account the information from the
non-missing elements. The Nuclear Norm, from other matrix norms, is
appropriate here because it makes the problem a convex optimization
problem (see discussion in the paper, page. 14).

    \hypertarget{estimation-procedure}{%
\section{Estimation procedure}\label{estimation-procedure}}

    Athey et al.~propose an iterative algorithm which may seem weird at
first as a solution to the problem. But before we go there, notice the
following: - In principle, for a complete matrix \(\mathbf{Y}\), it
holds that
\(\mathbf{Y} = \mathbf{P}_{\mathcal{O}}(\mathbf{Y}) + \mathbf{P}_{\mathcal{O}}^{\bot}(\mathbf{Y})\).
- However, we do not have the values for the opperation
\(\mathbf{P}_{\mathcal{O}}^{\bot}(\mathbf{Y})\) to work. - Assuming that
we had some \(\mathbf{L}_k\) that approximates the missing values of
\(\mathbf{Y}\) we could have
\(\mathbf{P}_{\mathcal{O}}^{\bot}(\mathbf{L}_k)\) in lieu. - But we want
to use the information in \(\mathbf{P}_{\mathcal{O}}(\mathbf{Y})\) to
generate the approximation \(\mathbf{L}_k\). The way to do this is by
the approximation using the singular value decomposition. This is
carried by the function \(\mathrm{shrink}_\lambda(\mathbf{A})\). - To be
clear, \(\mathrm{shrink}_\lambda(\mathbf{A})\) is simply a function that
returns an approximated version of the matrix \(\mathbf{A}\) using a
singular value decomposition by taking only the singular values larger
or equal than \(\lambda\). In other words, if
\(\mathbf{A}=\mathbf{U}\boldsymbol{\Sigma}\mathbf{V}^T\) \[
\mathrm{shrink}_\lambda(\mathbf{A}) = \mathbf{U}\boldsymbol{\Sigma}_{\mathrm{reduced}}\mathbf{V}^T,
\] where \(\boldsymbol{\Sigma}_{\mathrm{reduced}}\) is equal to
\(\boldsymbol{\Sigma}\) except that all diagonal elements \(k\) less
than \(\sigma_k(\mathbf{A})<\lambda\) have been set to zero. - Now, if
we initialize \(\mathbf{L}_1\) with some value, we can have an
approximation for \(\mathbf{Y}\) because
\(\mathbf{L}_2 = \mathrm{shrink}_\lambda(\mathbf{P}_{\mathcal{O}}(\mathbf{Y}) + \mathbf{P}_{\mathcal{O}}^{\bot}(\mathbf{L}_1))\).
- Maybe if we repeat this many times, it will converge to some
reasonable solution such that
\(\widehat{\mathbf{L}}=\lim_{n\rightarrow\infty}\mathbf{L}_n\).

I suppose they realized of this by thinking backwards. In other words,
the solution \(\widehat{\mathbf{L}}\) should satisfy the relation, for a
given value of \(\lambda\):
\[\widehat{\mathbf{L}} = \mathrm{shrink}_\lambda(\mathbf{P}_{\mathcal{O}}(\mathbf{Y}) + \mathbf{P}_{\mathcal{O}}^{\bot}(\widehat{\mathbf{L}})).\]
This is what's behind the iterative estimation algorithm.

    \hypertarget{the-matrix-completion-with-nuclear-norm-minimization-estimator-mcnnm}{%
\subsubsection{The ``Matrix-Completion with Nuclear Norm Minimization''
estimator
(MCNNM)}\label{the-matrix-completion-with-nuclear-norm-minimization-estimator-mcnnm}}

    Formally, they write it like: \[
\mathbf{L}_{k+1}(\lambda, \mathcal{O})=\mathrm{shrink}_{\frac{\lambda |\mathcal{O}| }{2}}\left\{\mathbf{P}_{\mathcal{O}}(\mathbf{Y}) + \mathbf{P}_{\mathcal{O}}^{\bot}(\mathbf{L}_{k}(\lambda, \mathcal{O}))\right\}.
\]

    \hypertarget{cross-validation}{%
\subsubsection{Cross-validation}\label{cross-validation}}

    The approximation, and as a consequence the estimate, will depend on the
value of \(\lambda\). As a hyper-parameter, we will choose its value
through cross-validation.

Athey et al.~propose to use some sort of \(K\)-fold cross-validation,
and choose \(K\) such that
\[\left|\mathcal{O}_k\right|/\left|\mathcal{O}\right| = \left|\mathcal{O}\right|/(NT).\]
That is, such that the fraction of non-missing values in a given
validation set \(\mathcal{O}_k\) is equal to the fraction of non-missing
values in the original matrix. We will do this making use of Grid
Search.

    \hypertarget{python-implementation}{%
\section{Python implementation}\label{python-implementation}}

    \begin{Verbatim}[commandchars=\\\{\}]
{\color{incolor}In [{\color{incolor}24}]:} \PY{c+c1}{\PYZsh{} Let\PYZsq{}s begin with the P operators}
         \PY{k}{def} \PY{n+nf}{PO\PYZus{}operator}\PY{p}{(}\PY{n}{Amat}\PY{p}{,} \PY{n}{setO}\PY{p}{)}\PY{p}{:}
             \PY{l+s+sd}{\PYZdq{}\PYZdq{}\PYZdq{}}
         \PY{l+s+sd}{    Set all elements that are not in setO to 0. It assumes that setO}
         \PY{l+s+sd}{    comes from applying np.nonzero(A==condition) to a matrix.}
         \PY{l+s+sd}{    \PYZdq{}\PYZdq{}\PYZdq{}}
             \PY{n}{Anew} \PY{o}{=} \PY{n}{np}\PY{o}{.}\PY{n}{zeros\PYZus{}like}\PY{p}{(}\PY{n}{Amat}\PY{p}{)}
             \PY{n}{Anew}\PY{p}{[}\PY{n}{setO}\PY{p}{]} \PY{o}{=} \PY{n}{Amat}\PY{p}{[}\PY{n}{setO}\PY{p}{]}
         
             \PY{k}{return} \PY{n}{Anew}
         
         \PY{k}{def} \PY{n+nf}{POcomp\PYZus{}operator}\PY{p}{(}\PY{n}{Amat}\PY{p}{,} \PY{n}{setO}\PY{p}{)}\PY{p}{:}
             \PY{l+s+sd}{\PYZdq{}\PYZdq{}\PYZdq{}}
         \PY{l+s+sd}{    The complement of PO\PYZus{}operator.}
         \PY{l+s+sd}{    Set all elements that are in setO to 0. It assumes that setO}
         \PY{l+s+sd}{    comes from applying np.nonzero(A==condition) to a matrix.}
         \PY{l+s+sd}{    \PYZdq{}\PYZdq{}\PYZdq{}}
             \PY{n}{Anew} \PY{o}{=} \PY{n}{np}\PY{o}{.}\PY{n}{copy}\PY{p}{(}\PY{n}{Amat}\PY{p}{)}
             \PY{n}{Anew}\PY{p}{[}\PY{n}{setO}\PY{p}{]} \PY{o}{=} \PY{l+m+mi}{0}
         
             \PY{k}{return} \PY{n}{Anew}
\end{Verbatim}


    \begin{Verbatim}[commandchars=\\\{\}]
{\color{incolor}In [{\color{incolor}37}]:} \PY{c+c1}{\PYZsh{} Lets code the shrink operator}
         \PY{k}{def} \PY{n+nf}{shrink}\PY{p}{(}\PY{n}{Amat}\PY{p}{,} \PY{n}{lamb}\PY{o}{=}\PY{l+m+mi}{0}\PY{p}{,} \PY{n}{doprint}\PY{o}{=}\PY{k+kc}{False}\PY{p}{)}\PY{p}{:}
             \PY{l+s+sd}{\PYZdq{}\PYZdq{}\PYZdq{}}
         \PY{l+s+sd}{    This generates a reduced version of A given by the singular value decomposition.}
         \PY{l+s+sd}{    It only takes the singular above lamb.}
         \PY{l+s+sd}{    \PYZdq{}\PYZdq{}\PYZdq{}}
             \PY{n}{U}\PY{p}{,} \PY{n}{Sigma}\PY{p}{,} \PY{n}{VT} \PY{o}{=} \PY{n}{np}\PY{o}{.}\PY{n}{linalg}\PY{o}{.}\PY{n}{svd}\PY{p}{(}\PY{n}{Amat}\PY{p}{,} \PY{n}{full\PYZus{}matrices}\PY{o}{=}\PY{k+kc}{False}\PY{p}{)}
             
             \PY{k}{if}\PY{p}{(}\PY{n}{doprint}\PY{p}{)}\PY{p}{:} \PY{n+nb}{print}\PY{p}{(}\PY{n}{Sigma}\PY{p}{)}
             
             \PY{n}{Sigma}\PY{p}{[}\PY{n}{Sigma} \PY{o}{\PYZlt{}} \PY{n}{lamb}\PY{p}{]} \PY{o}{=} \PY{l+m+mi}{0}
             
             \PY{k}{return} \PY{n}{U}\PY{n+nd}{@np}\PY{o}{.}\PY{n}{diag}\PY{p}{(}\PY{n}{Sigma}\PY{p}{)}\PY{n+nd}{@VT}
\end{Verbatim}


    \begin{Verbatim}[commandchars=\\\{\}]
{\color{incolor}In [{\color{incolor}143}]:} \PY{c+c1}{\PYZsh{} The loss function}
          \PY{k}{def} \PY{n+nf}{loss}\PY{p}{(}\PY{n}{Ymat}\PY{p}{,} \PY{n}{Lmat}\PY{p}{,} \PY{n}{setO}\PY{p}{,} \PY{n}{doprint}\PY{o}{=}\PY{k+kc}{False}\PY{p}{)}\PY{p}{:}
              \PY{n}{Ocardinality} \PY{o}{=} \PY{n+nb}{len}\PY{p}{(}\PY{n}{setO}\PY{p}{[}\PY{l+m+mi}{0}\PY{p}{]}\PY{p}{)}
              \PY{n}{diffmat} \PY{o}{=} \PY{n}{Ymat} \PY{o}{\PYZhy{}} \PY{n}{Lmat}
              \PY{k}{if}\PY{p}{(}\PY{n}{doprint}\PY{p}{)}\PY{p}{:} \PY{n+nb}{print}\PY{p}{(}\PY{n}{diffmat}\PY{p}{)}
              \PY{n}{outmat} \PY{o}{=} \PY{n}{PO\PYZus{}operator}\PY{p}{(}\PY{n}{diffmat}\PY{p}{,} \PY{n}{setO}\PY{p}{)}\PY{o}{*}\PY{o}{*}\PY{l+m+mi}{2}
              \PY{k}{return} \PY{p}{(}\PY{n}{outmat}\PY{o}{.}\PY{n}{sum}\PY{p}{(}\PY{p}{)}\PY{o}{.}\PY{n}{sum}\PY{p}{(}\PY{p}{)}\PY{o}{/}\PY{n}{Ocardinality}\PY{p}{)}\PY{o}{*}\PY{o}{*}\PY{l+m+mf}{0.5}
\end{Verbatim}


    \begin{Verbatim}[commandchars=\\\{\}]
{\color{incolor}In [{\color{incolor}161}]:} \PY{c+c1}{\PYZsh{} The \PYZdq{}Matrix\PYZhy{}Completion with Nuclear Norm Minimization\PYZdq{} estimator}
          \PY{k}{def} \PY{n+nf}{do\PYZus{}MCNNM}\PY{p}{(}\PY{n}{Ymat}\PY{p}{,} \PY{n}{setOk}\PY{p}{,} \PY{n}{lamb}\PY{o}{=}\PY{l+m+mi}{0}\PY{p}{,} \PY{n}{epsilon} \PY{o}{=} \PY{l+m+mf}{0.001}\PY{p}{,} \PY{n}{max\PYZus{}iters}\PY{o}{=}\PY{l+m+mi}{100}\PY{p}{,} \PY{n}{doprint}\PY{o}{=}\PY{k+kc}{False}\PY{p}{,} \PY{n}{printbatch} \PY{o}{=} \PY{l+m+mi}{10}\PY{p}{)}\PY{p}{:}
              \PY{l+s+sd}{\PYZdq{}\PYZdq{}\PYZdq{}}
          \PY{l+s+sd}{    This implements the iterative procedure to estimate L.}
          \PY{l+s+sd}{    \PYZdq{}\PYZdq{}\PYZdq{}}
              \PY{n}{N}\PY{p}{,} \PY{n}{T} \PY{o}{=} \PY{n}{Ymat}\PY{o}{.}\PY{n}{shape}
              
              \PY{c+c1}{\PYZsh{} Initialize L to the observed (non\PYZhy{}missing) values of Y given by the set setOk}
              \PY{n}{Lprev} \PY{o}{=} \PY{n}{PO\PYZus{}operator}\PY{p}{(}\PY{n}{Ymat}\PY{p}{,} \PY{n}{setOk}\PY{p}{)}
              
              \PY{c+c1}{\PYZsh{} Initialization of error with a highvalue and the iteration}
              \PY{n}{error} \PY{o}{=} \PY{n}{N}\PY{o}{*}\PY{n}{T}\PY{o}{*}\PY{l+m+mi}{10}\PY{o}{*}\PY{o}{*}\PY{l+m+mi}{3}
              \PY{n}{iteration} \PY{o}{=} \PY{l+m+mi}{0}
              
              \PY{k}{while}\PY{p}{(}\PY{p}{(}\PY{n}{error} \PY{o}{\PYZgt{}} \PY{n}{epsilon}\PY{p}{)} \PY{o+ow}{and} \PY{p}{(}\PY{n}{iteration} \PY{o}{\PYZlt{}} \PY{n}{max\PYZus{}iters}\PY{p}{)}\PY{p}{)}\PY{p}{:}
                  \PY{n}{Lnext} \PY{o}{=} \PY{n}{shrink}\PY{p}{(}\PY{n}{PO\PYZus{}operator}\PY{p}{(}\PY{n}{Ymat}\PY{p}{,} \PY{n}{setOk}\PY{p}{)} \PY{o}{+} \PY{n}{POcomp\PYZus{}operator}\PY{p}{(}\PY{n}{Lprev}\PY{p}{,} \PY{n}{setOk}\PY{p}{)}\PY{p}{,} \PY{n}{lamb} \PY{o}{=} \PY{n}{lamb}\PY{p}{)}
                  
                  \PY{c+c1}{\PYZsh{} Updating values}
                  \PY{n}{Lprev} \PY{o}{=} \PY{n}{Lnext}\PY{o}{.}\PY{n}{copy}\PY{p}{(}\PY{p}{)}
                  \PY{n}{error} \PY{o}{=} \PY{n}{loss}\PY{p}{(}\PY{n}{Ymat}\PY{p}{,} \PY{n}{Lprev}\PY{p}{,} \PY{n}{setOk}\PY{p}{)}
                  \PY{n}{iteration} \PY{o}{=} \PY{n}{iteration} \PY{o}{+} \PY{l+m+mi}{1}
                  
                  \PY{k}{if}\PY{p}{(}\PY{n}{doprint} \PY{o+ow}{and} \PY{p}{(}\PY{n}{iteration}\PY{o}{\PYZpc{}}\PY{k}{printbatch}==0 or iteration==1)):
                      \PY{n+nb}{print}\PY{p}{(}\PY{l+s+s2}{\PYZdq{}}\PY{l+s+s2}{Iteration }\PY{l+s+si}{\PYZob{}\PYZcb{}}\PY{l+s+se}{\PYZbs{}t}\PY{l+s+s2}{ Current loss: }\PY{l+s+si}{\PYZob{}\PYZcb{}}\PY{l+s+s2}{\PYZdq{}}\PY{o}{.}\PY{n}{format}\PY{p}{(}\PY{n}{iteration}\PY{p}{,} \PY{n}{error}\PY{p}{)}\PY{p}{)}
              
              \PY{k}{if}\PY{p}{(}\PY{n}{doprint}\PY{p}{)}\PY{p}{:}
                  \PY{n+nb}{print}\PY{p}{(}\PY{l+s+s2}{\PYZdq{}}\PY{l+s+s2}{\PYZdq{}}\PY{p}{)}
                  \PY{n+nb}{print}\PY{p}{(}\PY{l+s+s2}{\PYZdq{}}\PY{l+s+s2}{Final values:}\PY{l+s+s2}{\PYZdq{}}\PY{p}{)}
                  \PY{n+nb}{print}\PY{p}{(}\PY{l+s+s2}{\PYZdq{}}\PY{l+s+s2}{Iteration }\PY{l+s+si}{\PYZob{}\PYZcb{}}\PY{l+s+se}{\PYZbs{}t}\PY{l+s+s2}{ Current loss: }\PY{l+s+si}{\PYZob{}\PYZcb{}}\PY{l+s+s2}{\PYZdq{}}\PY{o}{.}\PY{n}{format}\PY{p}{(}\PY{n}{iteration}\PY{p}{,} \PY{n}{error}\PY{p}{)}\PY{p}{)}
                  \PY{n+nb}{print}\PY{p}{(}\PY{l+s+s2}{\PYZdq{}}\PY{l+s+s2}{\PYZdq{}}\PY{p}{)}
                  \PY{n+nb}{print}\PY{p}{(}\PY{n}{Ymat}\PY{p}{)}
                  \PY{n+nb}{print}\PY{p}{(}\PY{n}{np}\PY{o}{.}\PY{n}{round}\PY{p}{(}\PY{n}{Lnext}\PY{p}{,} \PY{l+m+mi}{2}\PY{p}{)}\PY{p}{)}
          
              \PY{k}{return} \PY{n}{Lnext}
\end{Verbatim}


    \begin{Verbatim}[commandchars=\\\{\}]
{\color{incolor}In [{\color{incolor} }]:} \PY{c+c1}{\PYZsh{} Cross\PYZhy{}validation}
        \PY{k}{def} 
\end{Verbatim}


    \hypertarget{toy-testing}{%
\subsection{Toy testing}\label{toy-testing}}

    \begin{Verbatim}[commandchars=\\\{\}]
{\color{incolor}In [{\color{incolor}145}]:} \PY{n}{A} \PY{o}{=} \PY{n}{np}\PY{o}{.}\PY{n}{array}\PY{p}{(}\PY{p}{[}\PY{p}{[}\PY{l+m+mi}{1}\PY{p}{,}\PY{l+m+mi}{2}\PY{p}{,}\PY{n}{np}\PY{o}{.}\PY{n}{nan}\PY{p}{,} \PY{l+m+mi}{0}\PY{p}{]}\PY{p}{,}\PY{p}{[}\PY{n}{np}\PY{o}{.}\PY{n}{nan}\PY{p}{,}\PY{l+m+mi}{2}\PY{p}{,}\PY{l+m+mi}{2}\PY{p}{,}\PY{l+m+mi}{1}\PY{p}{]}\PY{p}{,}\PY{p}{[}\PY{l+m+mi}{3}\PY{p}{,}\PY{l+m+mi}{0}\PY{p}{,}\PY{l+m+mi}{1}\PY{p}{,}\PY{l+m+mi}{3}\PY{p}{]}\PY{p}{,}\PY{p}{[}\PY{l+m+mi}{0}\PY{p}{,}\PY{l+m+mi}{0}\PY{p}{,}\PY{l+m+mi}{2}\PY{p}{,}\PY{n}{np}\PY{o}{.}\PY{n}{nan}\PY{p}{]}\PY{p}{]}\PY{p}{)}
\end{Verbatim}


    \begin{Verbatim}[commandchars=\\\{\}]
{\color{incolor}In [{\color{incolor}128}]:} \PY{n}{A}
\end{Verbatim}


\begin{Verbatim}[commandchars=\\\{\}]
{\color{outcolor}Out[{\color{outcolor}128}]:} array([[  1.,   2.,  nan,   0.],
                 [ nan,   2.,   2.,   1.],
                 [  3.,   0.,   1.,   3.],
                 [  0.,   0.,   2.,  nan]])
\end{Verbatim}
            
    \begin{Verbatim}[commandchars=\\\{\}]
{\color{incolor}In [{\color{incolor}129}]:} \PY{n}{observedset} \PY{o}{=} \PY{n}{np}\PY{o}{.}\PY{n}{nonzero}\PY{p}{(}\PY{o}{\PYZti{}}\PY{n}{np}\PY{o}{.}\PY{n}{isnan}\PY{p}{(}\PY{n}{A}\PY{p}{)}\PY{p}{)}
          \PY{n+nb}{print}\PY{p}{(}\PY{n+nb}{len}\PY{p}{(}\PY{n}{observedset}\PY{p}{[}\PY{l+m+mi}{0}\PY{p}{]}\PY{p}{)}\PY{p}{)}
          \PY{n}{observedset}
\end{Verbatim}


    \begin{Verbatim}[commandchars=\\\{\}]
13

    \end{Verbatim}

\begin{Verbatim}[commandchars=\\\{\}]
{\color{outcolor}Out[{\color{outcolor}129}]:} (array([0, 0, 0, 1, 1, 1, 2, 2, 2, 2, 3, 3, 3], dtype=int64),
           array([0, 1, 3, 1, 2, 3, 0, 1, 2, 3, 0, 1, 2], dtype=int64))
\end{Verbatim}
            
    \begin{Verbatim}[commandchars=\\\{\}]
{\color{incolor}In [{\color{incolor}130}]:} \PY{n}{PO\PYZus{}operator}\PY{p}{(}\PY{n}{A}\PY{p}{,} \PY{n}{observedset}\PY{p}{)}
\end{Verbatim}


\begin{Verbatim}[commandchars=\\\{\}]
{\color{outcolor}Out[{\color{outcolor}130}]:} array([[ 1.,  2.,  0.,  0.],
                 [ 0.,  2.,  2.,  1.],
                 [ 3.,  0.,  1.,  3.],
                 [ 0.,  0.,  2.,  0.]])
\end{Verbatim}
            
    \begin{Verbatim}[commandchars=\\\{\}]
{\color{incolor}In [{\color{incolor}131}]:} \PY{n}{POcomp\PYZus{}operator}\PY{p}{(}\PY{n}{A}\PY{p}{,} \PY{n}{observedset}\PY{p}{)}
\end{Verbatim}


\begin{Verbatim}[commandchars=\\\{\}]
{\color{outcolor}Out[{\color{outcolor}131}]:} array([[  0.,   0.,  nan,   0.],
                 [ nan,   0.,   0.,   0.],
                 [  0.,   0.,   0.,   0.],
                 [  0.,   0.,   0.,  nan]])
\end{Verbatim}
            
    \begin{Verbatim}[commandchars=\\\{\}]
{\color{incolor}In [{\color{incolor}132}]:} \PY{n}{PO\PYZus{}operator}\PY{p}{(}\PY{n}{A}\PY{p}{,} \PY{n}{observedset}\PY{p}{)} \PY{o}{+} \PY{n}{POcomp\PYZus{}operator}\PY{p}{(}\PY{n}{A}\PY{p}{,} \PY{n}{observedset}\PY{p}{)}
\end{Verbatim}


\begin{Verbatim}[commandchars=\\\{\}]
{\color{outcolor}Out[{\color{outcolor}132}]:} array([[  1.,   2.,  nan,   0.],
                 [ nan,   2.,   2.,   1.],
                 [  3.,   0.,   1.,   3.],
                 [  0.,   0.,   2.,  nan]])
\end{Verbatim}
            
    \begin{Verbatim}[commandchars=\\\{\}]
{\color{incolor}In [{\color{incolor}133}]:} \PY{n}{shrink}\PY{p}{(}\PY{n}{PO\PYZus{}operator}\PY{p}{(}\PY{n}{A}\PY{p}{,} \PY{n}{observedset}\PY{p}{)}\PY{p}{,} \PY{n}{doprint}\PY{o}{=}\PY{k+kc}{True}\PY{p}{)}
\end{Verbatim}


    \begin{Verbatim}[commandchars=\\\{\}]
[ 4.77539086  3.01152294  2.12336094  0.78594529]

    \end{Verbatim}

\begin{Verbatim}[commandchars=\\\{\}]
{\color{outcolor}Out[{\color{outcolor}133}]:} array([[  1.00000000e+00,   2.00000000e+00,   2.22044605e-16,
                   -2.22044605e-16],
                 [  0.00000000e+00,   2.00000000e+00,   2.00000000e+00,
                    1.00000000e+00],
                 [  3.00000000e+00,   0.00000000e+00,   1.00000000e+00,
                    3.00000000e+00],
                 [  6.10622664e-16,   8.88178420e-16,   2.00000000e+00,
                   -7.77156117e-16]])
\end{Verbatim}
            
    \begin{Verbatim}[commandchars=\\\{\}]
{\color{incolor}In [{\color{incolor}134}]:} \PY{n}{L1} \PY{o}{=} \PY{n}{shrink}\PY{p}{(}\PY{n}{PO\PYZus{}operator}\PY{p}{(}\PY{n}{A}\PY{p}{,} \PY{n}{observedset}\PY{p}{)}\PY{p}{,} \PY{n}{lamb} \PY{o}{=} \PY{l+m+mf}{1.5}\PY{p}{)}
          \PY{n}{L1}
\end{Verbatim}


\begin{Verbatim}[commandchars=\\\{\}]
{\color{outcolor}Out[{\color{outcolor}134}]:} array([[  7.46799657e-01,   2.04660961e+00,  -1.03464140e-01,
                    2.88085806e-01],
                 [  2.75984114e-01,   1.94919631e+00,   2.11277417e+00,
                    6.85991318e-01],
                 [  3.00188696e+00,  -3.47354839e-04,   1.00077106e+00,
                    2.99785306e+00],
                 [ -3.27503074e-01,   6.02873970e-02,   1.86617386e+00,
                    3.72625827e-01]])
\end{Verbatim}
            
    \begin{Verbatim}[commandchars=\\\{\}]
{\color{incolor}In [{\color{incolor}135}]:} \PY{n}{loss}\PY{p}{(}\PY{n}{A}\PY{p}{,} \PY{n}{L1}\PY{p}{,} \PY{n}{observedset}\PY{p}{,} \PY{n}{doprint}\PY{o}{=}\PY{k+kc}{True}\PY{p}{)}
\end{Verbatim}


    \begin{Verbatim}[commandchars=\\\{\}]
[[ 0.25320034 -0.04660961         nan -0.28808581]
 [        nan  0.05080369 -0.11277417  0.31400868]
 [-0.00188696  0.00034735 -0.00077106  0.00214694]
 [ 0.32750307 -0.0602874   0.13382614         nan]]

    \end{Verbatim}

\begin{Verbatim}[commandchars=\\\{\}]
{\color{outcolor}Out[{\color{outcolor}135}]:} 0.043955478446319037
\end{Verbatim}
            
    \begin{Verbatim}[commandchars=\\\{\}]
{\color{incolor}In [{\color{incolor}148}]:} \PY{n}{Lest} \PY{o}{=} \PY{n}{do\PYZus{}MCNNM}\PY{p}{(}\PY{n}{A}\PY{p}{,} \PY{n}{observedset}\PY{p}{,} \PY{n}{lamb} \PY{o}{=} \PY{l+m+mf}{2.5}\PY{p}{,} \PY{n}{epsilon}\PY{o}{=}\PY{l+m+mi}{10}\PY{o}{*}\PY{o}{*}\PY{p}{(}\PY{o}{\PYZhy{}}\PY{l+m+mi}{6}\PY{p}{)}\PY{p}{,} \PY{n}{doprint}\PY{o}{=}\PY{k+kc}{True}\PY{p}{)}
\end{Verbatim}


    \begin{Verbatim}[commandchars=\\\{\}]
Iteration 1	 Current loss: 0.5529274516247817
Iteration 10	 Current loss: 0.37461275410122796
Iteration 20	 Current loss: 0.3698980768609287
Iteration 30	 Current loss: 0.3694973444179368
Iteration 40	 Current loss: 0.36943523600957867
Iteration 50	 Current loss: 0.36942435600321566
Iteration 60	 Current loss: 0.36942236604808826
Iteration 70	 Current loss: 0.3694219954876393
Iteration 80	 Current loss: 0.369421925951333
Iteration 90	 Current loss: 0.3694219128593888
Iteration 100	 Current loss: 0.36942191039096084

Final values:
Iteration 100	 Current loss: 0.36942191039096084

[[  1.   2.  nan   0.]
 [ nan   2.   2.   1.]
 [  3.   0.   1.   3.]
 [  0.   0.   2.  nan]]
[[ 0.98  2.01  3.49  0.02]
 [ 1.56  1.33  2.39  0.97]
 [ 3.01  0.33  0.81  3.01]
 [ 0.04  0.87  1.49 -0.4 ]]

    \end{Verbatim}

    \hypertarget{using-the-abadie-diamond-hainmueller-california-smoking-data}{%
\section{Using the Abadie-Diamond-Hainmueller California Smoking
Data}\label{using-the-abadie-diamond-hainmueller-california-smoking-data}}

    \begin{Verbatim}[commandchars=\\\{\}]
{\color{incolor}In [{\color{incolor}153}]:} \PY{n}{PATH\PYZus{}FROM\PYZus{}URL} \PY{o}{=} \PY{l+s+s2}{\PYZdq{}}\PY{l+s+s2}{https://raw.githubusercontent.com/susanathey/MCPanel/master/tests/examples\PYZus{}from\PYZus{}paper/california/}\PY{l+s+s2}{\PYZdq{}}
          \PY{n}{mydf} \PY{o}{=} \PY{n}{pd}\PY{o}{.}\PY{n}{read\PYZus{}csv}\PY{p}{(}\PY{n}{PATH\PYZus{}FROM\PYZus{}URL} \PY{o}{+} \PY{l+s+s2}{\PYZdq{}}\PY{l+s+s2}{smok\PYZus{}outcome.csv}\PY{l+s+s2}{\PYZdq{}}\PY{p}{,} \PY{n}{header}\PY{o}{=}\PY{k+kc}{None}\PY{p}{)}\PY{o}{.}\PY{n}{values}\PY{o}{.}\PY{n}{T}
\end{Verbatim}


    \begin{Verbatim}[commandchars=\\\{\}]
{\color{incolor}In [{\color{incolor}154}]:} \PY{n+nb}{print}\PY{p}{(}\PY{n}{mydf}\PY{p}{)}
          \PY{n+nb}{print}\PY{p}{(}\PY{n}{mydf}\PY{o}{.}\PY{n}{shape}\PY{p}{)}
\end{Verbatim}


    \begin{Verbatim}[commandchars=\\\{\}]
[[ 123.   121.   123.5 {\ldots},   52.3   47.2   41.6]
 [  89.8   95.4  101.1 {\ldots},  106.2  100.7   96.2]
 [ 100.3  104.1  103.9 {\ldots},  109.5  104.8   99.4]
 {\ldots}, 
 [ 114.5  111.5  117.5 {\ldots},  114.6  112.4  107.9]
 [ 106.4  105.4  108.8 {\ldots},   88.7   84.4   80.1]
 [ 132.2  131.7  140.  {\ldots},  102.9  104.8   90.5]]
(39, 31)

    \end{Verbatim}

    \begin{Verbatim}[commandchars=\\\{\}]
{\color{incolor}In [{\color{incolor}160}]:} \PY{n}{plt}\PY{o}{.}\PY{n}{imshow}\PY{p}{(}\PY{n}{mydf}\PY{p}{,} \PY{n}{cmap}\PY{o}{=}\PY{n}{plt}\PY{o}{.}\PY{n}{cm}\PY{o}{.}\PY{n}{gray}\PY{p}{)}\PY{p}{;}
\end{Verbatim}


    \begin{center}
    \adjustimage{max size={0.9\linewidth}{0.9\paperheight}}{output_40_0.png}
    \end{center}
    { \hspace*{\fill} \\}
    
    \begin{Verbatim}[commandchars=\\\{\}]
{\color{incolor}In [{\color{incolor} }]:} \PY{n}{Y} \PY{o}{=} \PY{n}{Y}\PY{p}{[}\PY{l+m+mi}{1}\PY{p}{:}\PY{p}{,}\PY{p}{:}\PY{p}{]}  \PY{c+c1}{\PYZsh{} drop the first row since it is treated and the untreated values for that unit are not available.}
        
        \PY{n}{N}\PY{p}{,} \PY{n}{T} \PY{o}{=} \PY{n}{Y}\PY{o}{.}\PY{n}{shape}
\end{Verbatim}



    % Add a bibliography block to the postdoc
    
    
    
    \end{document}
